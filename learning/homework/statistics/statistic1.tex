\documentclass[UTF8]{ctexart}
\usepackage{ctex,geometry}
\usepackage{mathrsfs}
\usepackage{amsmath}
\usepackage{amssymb}
\usepackage{booktabs}
\pagestyle{plain}
\geometry{left=2.5cm,right=2.5cm,top=2.0cm,bottom=2cm}
\begin{document}

\begin{center}
	\LARGE
	2019级心理多元统计课程作业\_第一次

	\normalsize
	谷菲 201928012503005
\end{center}

\paragraph{1 参数与统计量的英文与符号}
~\\
%$\left(\begin{tabular}{ccc|c}
%
%a11 & a12 & a13  & b1 \\
%a21 & a22  & a23 & b2  \\ 
%a31 & a32  & a33 & b3  \\
%
%\end{tabular}\right)$
\begin{tabular}{cccc}
    \toprule
    统计特征 & 英文 & 符号(参数) & 符号(统计量)\\
    \midrule
     算术平均数 & Arithmetic mean            &$\mu$ & $\Bar{X}$     \\
     方差       & Variance                   &$\sigma^2$ & $S^2$         \\
     标准差     & Standard deviation         &$\sigma$ & $S$           \\
     相关系数   & Correlation coefficient    &$\rho$ & $r$      \\
     回归系数   & Regression coefficient     &$\beta$ & $b$    \\
     \bottomrule
\end{tabular}
\\
\\

%其它统计特征及其英文
\begin{tabular}{cccc}
	\toprule
	统计特征 & 英文 & 统计特征 & 英文\\
	\midrule
	最大值   & Maximum          &众数 &Mode     \\
	最小值   & Minimum            	 &峰度 &Kurtosis         \\
	计数     & Count       &偏度 & Skewness           \\
	标准误   & Standard error &全距 & Range      \\
	中位数   & Median		\\
	\bottomrule
\end{tabular}
%最大值,最小值,计数,标准误,中位数,众数,峰度,偏度,全距

\paragraph{2 算数平均数的性质}
%\subparagraph{2 算数平均数的性质}
~\\

各变量值与其均值离差之和等于零

各变量值与其均值的离差平方和最小

两个正数的算数平均数总是大于等于几何平均数

易受极端值的影响

\paragraph{3 销售数据的计算}
~\\
(1) 销售人员汽车销售量的众数、中数、平均数

先将数据进行排序
1,2,2,3,3,3,3,4,4,4,4,4,5,5,6,7,7,7,8,10

可以发现5出现的频次最高,因此众数为5

中位数为第10位和第11位两个数的平均值,因此为4
~\\

平均值$\Bar{X}=\dfrac{1+2\times 2+3\times 4+4\times 5+5\times 2+6+7\times 3+8+10}{20}=4.6$
~\\


\noindent(2) 销售量的四分位差、平均差、方差、标准差

上四分位数为第15与第16项的平均值6.5

下四分位数为第5与第6项的平均值3

$\therefore$
四分位差$Q_D=6.75-3=3.5$
~\\

平均差$AD=\dfrac{\displaystyle \sum_{i=1}^{n}(X_i-\bar{X})}{n}=\dfrac{\displaystyle \sum_{i=1}^{20}(X_i-4.6)}{20}\dfrac{36.4}{20}=1.82$
~\\

方差$S^2=\dfrac{\displaystyle \sum_{i=1}^n (X_i-\Bar{X})^2}{n-1}=\dfrac{\displaystyle\sum_{i=1}^{20} (X_i-4.6)}{19}=\dfrac{98.8}{19}=5.2$
~\\

标准差$S=\sqrt{S^2}=\sqrt{5.2}\approx2.28$
~\\

%这里偏度和峰度得到的值和SPSS的结果并不一致,和R计算出来的差不多(moments的函数)
\noindent(3) 偏度、峰度

偏度$a_3= \dfrac{\displaystyle\sum_{i=1}^{n} (X_i-\Bar{X})^3}{nS^3}=\dfrac{\displaystyle \sum_{i=1}^{20} (X_i-4.6)^3}{20\times 2.28^3}\approx0.60$
~\\

峰度$a_4= \dfrac{\displaystyle \sum_{i=1}^{n} (X_i-\Bar{X})^4}{nS^4}\dfrac{\displaystyle \sum_{i=1}^{20} (X_i-4.6)^4}{20\times 5.2^2}\approx2.54$
\\

\noindent
使用SPSS计算所用到的syntax如下
\begin{verbatim}
FREQUENCIES VARIABLES=sale 
/NTILES=4 
/STATISTICS=STDDEV VARIANCE MEAN MEDIAN MODE SKEWNESS SESKEW KURTOSIS SEKURT 
/ORDER=ANALYSIS.
*计算平均差.
COMPUTE AD=ABS(sale - 4.6). 
EXECUTE.  
DESCRIPTIVES VARIABLES=AD 
/STATISTICS=MEAN.
\end{verbatim}

\paragraph{4 居民状态调查}
\begin{verbatim}
COMPUTE Toptim=(6-op2)+(6-op4)+(6-op6)+op1+op3+op5.
COMPUTE Tmast=(5-mast1)+(5-mast3)+(5-mast4)+(5-mast6)+(5-mast7)+mast2+mast5.
COMPUTE Tposaff=pn1+pn4+pn6+pn7+pn9+pn12+pn13+pn15+pn17+pn18.
COMPUTE Tnegaff=pn2+pn3+pn5+pn8+pn10+pn11+pn14+pn16+pn19+pn20.
COMPUTE Tlifesat=SUM(lifsat1 TO lifsat5).
COMPUTE Tpstress=(6-pss4)+(6-pss5)+(6-pss7)+(6-pss8)+pss1+pss2+pss3+pss6+pss9+pss10.
COMPUTE Tselfest=(5-sest3)+(5-sest5)+(5-sest7)+(5-sest9)+(5-sest10)+sest1+sest2+sest4+sest6+sest8.
COMPUTE Tmarlow=SUM(m1 TO m5)+(1-m6)+(1-m7)+(1-m8)+(1-m9)+(1-m10).
COMPUTE Tpcoiss=(6-pc1)+(6-pc2)+(6-pc7)+(6-pc11)+(6-pc15)+(6-pc16)+pc3+pc4+pc5+pc6+pc8+pc9+pc10+pc12+pc13+pc14+pc17+pc18.
EXECUTE.

RECODE educ (3=2) (4=3) (5=4) (6=5) (1 thru 2=1) INTO educ2.
RECODE age (18 thru 29=1) (30 thru 44=2) (45 thru Highest=3) INTO Agegp3.
RECODE age (18 thru 24=1) (25 thru 32=2) (33 thru 40=3) (41 thru 49=4) (50 thru Highest=5) INTO Agegp5.
EXECUTE.

FREQUENCIES VARIABLES=Tlifesat
/NTILES=4
/PERCENTILES=80.0 
/STATISTICS=STDDEV VARIANCE MEAN MEDIAN MODE
/ORDER=ANALYSIS.

EXAMINE VARIABLES=Tpstress
/PLOT BOXPLOT STEMLEAF HISTOGRAM
/COMPARE VARIABLES
/STATISTICS NONE
/CINTERVAL 95
/MISSING LISTWISE
/NOTOTAL.

EXAMINE VARIABLES=Tposaff Tnegaff Tselfest BY sex
/PLOT NONE
/STATISTICS DESCRIPTIVES
/CINTERVAL 95
/MISSING LISTWISE
/NOTOTAL.
\end{verbatim}

\end{document}
