\documentclass[UTF8]{ctexart}
\usepackage{ctex,geometry}
\usepackage{mathrsfs}
\usepackage{amsmath}
\usepackage{amssymb}
\usepackage{booktabs}
\pagestyle{plain}
\geometry{left=2.5cm,right=2.5cm,top=2.0cm,bottom=2cm}
\begin{document}

\begin{center}
	\LARGE
	2019级心理多元统计课程作业\_第二次

	\normalsize
	谷菲 201928012503005
\end{center}

\paragraph{1}(1)该应试者哪项测试更为理想?
~\\

$Z_A=\dfrac{115-100}{15}=1  \qquad $
$Z_B=\dfrac{425-400}{50}=0.5$
~\\

$\therefore$在A项测试中分数更为理想
~\\

(2)哪项测试的离散程度更大? 

因为B的标准差更大,所以其离散程度更大

\paragraph{2一个总体有10个数据,平均数μ=9,标准差σ=3,如果其中某个分数X=10从总体中去掉,那么现在总体的平均数和标准差分别是多少?}
~\\


$\mu'=\dfrac{n\mu-X}{n-1}=\dfrac{90-10}{9}=\dfrac{80}{9}$
	
$\because $
$
\sigma^2=\dfrac{\displaystyle \sum_{i=1}^{n}X_i^2}{n}-\mu^2
\qquad 
\sigma'^2=\dfrac{\displaystyle \sum_{i=1}^{n-1}X_i^2}{n-1}-\mu'^2
$

$ \qquad \displaystyle\sum_{i=1}^{n-1}X_i^2=\sum_{i=1}^{n}X_i^2-X^2=\sum_{i=1}^{n}X_i^2-100$
~\\

$\sigma'=\sqrt{\dfrac{n(\sigma^2+\mu^2)-100}{n-1}-\mu'^2}=\sqrt{\dfrac{800}{9}-(\dfrac{80}{9})^2}\approx3.14$

\paragraph{3相关的适用情况}
~\\

\begin{tabular}{ll}
	\toprule
	数据情况 & 相关方法 \\
	\midrule
	两变量等距以上但非正态分布 & 等级相关           \\
	两变量等距以上且正态分布     & 积差相关      \\
	两变量等距以上、正态,都被认为化为两类   & 四分相关\\
	两变量等距以上,但其中一个变量被认为化为两类   &二列相关\\
	两变量等距以上,但其中一个变量被认为化为多类	&多系列相关\\
	一个变量为等距以上且正态发表,另一个变量为二分变量	&点二列相关\\
	两个变量都为二分变量	&$\phi$相关系数\\
	两个变量都为顺序变量	&列联相关系数\\
	两个变量都为两类的类别变量 &$\phi$相关系数\\
	\bottomrule
%	$\sqrt{\dfrac{SS}{n(n-1)}}$	
\end{tabular}
\newpage
\paragraph{4斯皮尔曼相关}
~\\

%\begin{center}
	\begin{tabular}{ccccc}
		\toprule
		排名 & 成绩 & 成绩的等级 & $D_i$ & $D_i^2$ \\
		\midrule
		1 & A & 1.5 & -0.5 & 0.25 \\
		2 & B & 4 & -2 & 4 \\
		3 & A & 1.5 & 1.5 & 2.25 \\
		4 & B & 4 & 0 & 0 \\
		5 & B & 4 & 1 & 1 \\
		6 & C & 7 & -1 & 1 \\
		7 & D & 9.5 & -2.5 & 6.25 \\
		8 & C & 7 & 1 & 1 \\
		9 & C & 7 & 2 & 4 \\
		10 & D & 9.5 & 0.5 & 0.25 \\
		11 & E & 11 & 0 & 0 \\
		\bottomrule
	\end{tabular}
%\end{center}

~\\

%$r_k=1-\dfrac{\displaystyle 6\sum_{i=1}^{n}D_i^2}{n(n^2-1)}$
$\because$排名没有重复的
\\

$\therefore \sum x^2=\dfrac{n(n^2-1)}{12}=110$
\\

$\because$成绩有重复的,重复的数量分别为2,2,3,3
\\

$\therefore \sum y^2=\dfrac{n(n^2-1)}{12}-\left(\dfrac{2\times(2^2-1)}{12}+\dfrac{2\times(2^2-1)}{12}+\dfrac{3\times(2^3-1)}{12}+\dfrac{3\times (3^2-1)}{12}\right)=105$
\\

$r_R=\dfrac{\sum x^2+\sum y^2 -\sum D^2}{2\sqrt{\sum X^2\sum y^2}}=\dfrac{110+105-20}{2\times \sqrt{110\times 105}}\approx0.91$
\\

当$\alpha=0.05$,自由度为9时,斯皮尔曼相关的临界值是0.7

$\therefore$相关系数是显著的,两人的评分具有一致性
\\

\noindent
使用SPSS计算所用到的syntax如下
\begin{verbatim}
	NONPAR CORR 
	/VARIABLES=rank score 
	/PRINT=SPEARMAN TWOTAIL NOSIG 
	/MISSING=PAIRWISE.
\end{verbatim}

\newpage
\paragraph{5皮尔逊相关}
~\\

\begin{tabular}{ccccccc}
\toprule
焦虑等级(X) & 考试分数(Y) & $X-\bar{X}$ & $Y-\bar{Y}$ & $(X-\bar{X})^2$ & $(Y-\bar{Y})^2$ & $(X-\bar{X})(Y-\bar{Y})$ \\
\midrule
5 & 80 & 0 & -3 & 0 & 9 & 0 \\
2 & 88 & -3 & 5 & 9 & 25 & -15 \\
7 & 80 & 2 & -3 & 4 & 9 & -6 \\
7 & 79 & 2 & -4 & 4 & 16 & -8 \\
4 & 86 & -1 & 3 & 1 & 9 & -3 \\
5 & 85 & 0 & 2 & 0 & 4 & 0 \\

\bottomrule
\end{tabular}
~\\

$r=\dfrac{\displaystyle \sum_{i=1}^{n}(X_i-\bar{X})(Y_i-\bar{Y})}{\sqrt{\displaystyle \sum_{i=1}^{n}(X_i-\bar{X})^2\sum_{i=1}^{n}(Y_i-\bar{Y})^2}}=\dfrac{-32}{\sqrt{18\times 72}}\approx-0.89$
~\\

$df=n-2=6-2=4$

当$\alpha=0.05$时,相关系数的临界值是$r_\alpha=0.811$

$\because r>0.811$

$\therefore$焦虑与考试分数之间有显著的相关关系

\noindent
使用SPSS计算所用到的syntax如下
\begin{verbatim}
CORRELATIONS 
/VARIABLES=stress examscore 
/PRINT=TWOTAIL NOSIG 
/MISSING=PAIRWISE.
\end{verbatim}


\paragraph{6点二列相关}
~\\

\begin{tabular}{cc}
\toprule
分数 & 训练 \\
\midrule
4 & 0 \\
7 & 0 \\
3 & 0 \\
6 & 0 \\
9 & 1 \\
7 & 1 \\
6 & 1 \\
10 & 1 \\

\bottomrule
\end{tabular}

经过训练的平均值$\bar{X_p}=\dfrac{9+7+6+10}{4}=8$

没有训练的平均值$\bar{X_q}=\dfrac{4+7+3+6}{4}=5$

经过训练的比例$p=0.5$

没有训练的比例$q=0.5$

$S_t=\sqrt{\dfrac{\displaystyle \sum_{i=1}^n (X_i-\Bar{X})^2}{n-1}}\approx2.33$
~\\

$r_pb=\dfrac{\bar{X_p}-\bar{X_q}}{S_t}\sqrt{pq}=0.5\times\sqrt{\dfrac{8-5}{2.33}}\approx0.57$
~\\

\noindent
使用SPSS计算所用到的syntax如下
\begin{verbatim}
NONPAR CORR 
/VARIABLES=score train 
/PRINT=SPEARMAN TWOTAIL NOSIG 
/MISSING=PAIRWISE.
\end{verbatim}


\paragraph{7计算$\varphi$系数}
~\\

原始的数据形式

\begin{tabular}{cc}
\toprule
年龄 & 喜欢A \\
\midrule
0 & 1 \\
0 & 0 \\
0 & 0 \\
0 & 0 \\
1 & 1 \\
1 & 1 \\
1 & 1 \\
1 & 1 \\
1 & 1 \\
1 & 0 \\
1 & 0 \\
1 & 0 \\

\bottomrule
\end{tabular}

~\\

转换为四格表的形式

\begin{tabular}{ccc}
\toprule
 & 喜欢A & 不喜欢A \\
\midrule
30岁以上 & 5 & 3 \\
30岁以下 & 1 & 3 \\

\bottomrule
\end{tabular}

~\\

$\varphi=\dfrac{|ad-bc|}{\sqrt{(a+b)(c+d)(a+c)(b+d)}}=\dfrac{15-3}{\sqrt{8\times6\times6\times4}}\approx0.35$
~\\

\noindent
使用SPSS计算所用到的syntax如下
\begin{verbatim}
CROSSTABS 
/TABLES=age BY loveA 
/FORMAT=AVALUE TABLES 
/STATISTICS=PHI 
/CELLS=COUNT 
/COUNT ROUND CELL.
\end{verbatim}

\end{document}