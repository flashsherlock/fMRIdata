\documentclass[UTF8]{ctexart}
\usepackage{ctex,geometry}
\usepackage{mathrsfs}
\usepackage{amsmath}
\usepackage{amssymb}
\pagestyle{plain}
%\topmargin -1.5in
%\textheight 11in
%\oddsidemargin -.25in
%\evensidemargin -.25in
%\textwidth 7in
\geometry{left=2.5cm,right=2.5cm,top=2.0cm,bottom=2cm}
\begin{document}
%\author{谷菲 201928012503005}
%\title{2019级心理多元统计课程作业\_第三次}
%\maketitle
\begin{center}
	\LARGE
	2019级心理多元统计课程作业\_第三次

	\normalsize
	谷菲 201928012503005
\end{center}
\paragraph{1 两个骰子掷一次,出现两个相同点数的概率是多少?}
~\\

%\indent
两个骰子掷一次总共有$6\times6=36$种情况,其中两个点数相同的情况共有6种

$\therefore$
$$P(\mbox{两点数相同})=\dfrac{6}{36}=\dfrac{1}{6}$$

\paragraph{2 设A与B是两个随机事件,已知A与B至少有一个发生的概率是1/3,A发生且B不发生的概率是1/9,求B发生的概率?}
~\\

$ \because $
$$P(\mbox{A与B至少有一个发生})=P(A \cup B)=P(A)+P(B)-P(A \cap B)=P(A)+P(B)-P(A)\cdot P(B)=\dfrac{1}{3}$$
$$P(\mbox{A发生且B不发生})=P(A \cap \complement _UB)=P(A) \cdot P(\complement _UB)=P(A)\cdot (1-P(B))=P(A)-P(A)\cdot P(B)=\dfrac{1}{9}$$

$ \therefore $
$$P(B)=P(\mbox{A与B至少有一个发生})-P(\mbox{A发生且B不发生})=\dfrac{1}{3}-\dfrac{1}{9}=\dfrac{2}{9}$$

\paragraph{3 用E字型视标检查儿童的视敏度,每种视力值(1.0,1.5)有四个方向的E字(共8个),问:说对了几个才能说真看清了而不是猜对的?}
~\\

每个E字回答正确的概率为$ p=\dfrac{1}{4} $,答错的概率$ q=1-p=\dfrac{3}{4} $,答对的数量$ X $服从二项分布,均值为$ np=2 $,方差为$ npq=\dfrac{3}{2} $

$\because $
$$P(Z>1.645)=0.05$$

$ \therefore $
$$Z=\dfrac{X-np}{\sqrt{npq}}=\dfrac{\sqrt2(X-2)}{\sqrt{3}}>1.645$$

$$X>4.01$$

所以至少答对5个E字的方向才能说明是真看清了


\paragraph{4 今有四择一选择测验100题,问答对多少题才能说他的回答不是完全凭猜测?}
~\\

任意一道题目回答正确的概率为$ p=\dfrac{1}{4} $,答错的概率$ q=1-p=\dfrac{3}{4} $,答对的题目数量$ X $近似服从均值为$ np=25 $,方差为$ npq=\dfrac{75}{4} $

$\because $
$$P(Z>1.645)=0.05$$

$ \therefore $
$$Z=\dfrac{X-np}{\sqrt{npq}}=\dfrac{2(X-25)}{5\sqrt{3}}>1.645$$

$$X>32.12$$

所以至少答对33道才能说他的回答不是完全猜测的
%\clearpage

\paragraph{5 某篮球队员定点投篮的命中率为0.8,问该队员定点投篮10次,至少投中6次的概率是多少?}
~\\
$$
 \begin{aligned}
P(\mbox{至少投中6次})
 &=C_{10}^6\times 0.8^6 \times (1-0.8)^4+C_{10}^7\times 0.8^7 \times (1-0.8)^3+C_{10}^8\times 0.8^8 \times (1-0.8)^2\\
 &+C_{10}^9\times 0.8^9 \times (1-0.8)+C_{10}^10\times 0.8^{10}\\
 &\approx0.9672=96.72\%
\end{aligned}
$$

\paragraph{6 某企业职工的文化程度,小学占10\%,初中占50\%,高中及以上占40\%,25岁以下青年在上述各组中的比例分别为20\%、50\%、70\%,从该企业中随机抽取一名职工,发现其年龄在25岁以下,问他具有小学、初中、高中以上文化程度的概率各为多少?}
~\\

设随机抽取一名员工,其文化程度为小学、初中、高中及以上分别为事件$A_1, A_2, A_3$,年龄为25岁以下为事件$B$

$\because$
$$P(A_1)=0.1 \qquad P(A_2)=0.5 \qquad P(A_3)=0.4 $$
$$P(B\mid A_1)=0.2 \qquad P(B\mid A_2)=0.5 \qquad P(B\mid A_3)=0.7 $$
$ \therefore $
$$P(B)=P(B\mid A_1)\times P(A_1)+ P(B\mid A_2)\times P(A_2)+ P(B\mid A_3)\times P(A_3)=0.55 $$
$$P(A_1 \mid B)=\dfrac{P(B\mid A_1)\times P(A_1)}{P(B)}=\dfrac{0.02}{0.55}\approx3.64\% $$
$$P(A_2 \mid B)=\dfrac{P(B\mid A_2)\times P(A_2)}{P(B)}=\dfrac{0.25}{0.55}\approx45.45\%$$
$$P(A_3 \mid B)=\dfrac{P(B\mid A_3)\times P(A_3)}{P(B)}=\dfrac{0.28}{0.55}\approx50.91\%$$
\end{document}
