\documentclass[UTF8]{ctexart}
\usepackage{ctex,geometry}
\usepackage{mathrsfs}
\usepackage{amsmath}
\usepackage{amssymb}
% 插入三线表
\usepackage{booktabs}
% 行距的设置
\usepackage{setspace}
% 页码位于脚注
\pagestyle{plain}
\geometry{left=2.5cm,right=2.5cm,top=2.0cm,bottom=2cm}
\begin{document}

\begin{center}
	\LARGE
	2019级心理多元统计课程作业\_第四次

	\normalsize
	谷菲 201928012503005
\end{center}

\paragraph{1 已知某测验成绩的分布为正态分布,标准差为$\sigma=5$。从该总体中随机抽取$n=16$的样本,算得其平均值为81,标准差$S=6$,问该测验的真实分数是多少?}
~\\

%\begin{spacing}{2.0}
给定显著性水平为$\alpha=0.05$,则$Z_\frac{\alpha}{2}=1.96$

% $Z=\dfrac{\Bar{X}-\mu}{\dfrac{\sigma}{\sqrt{n}}}$
$\because$
$\varDelta=Z_\frac{\alpha}{2}\dfrac{\sigma}{\sqrt{n}}=1.96\times \dfrac{5}{\sqrt{16}}=2.45$

$\therefore$测验平均分数的95\%的置信区间是$81\pm2.45$ ,即$(78.55,83.45)$
%\end{spacing}


\paragraph{2 已知某校学生的身高服从正态分布,现从该校随机抽取20名学生测量身高,其平均值为171cm,标准差为6cm。试估计该校学生身高的真实情况。}
~\\

%\begin{spacing}{2.0}
给定显著性水平为$\alpha=0.05$,则$Z_\frac{\alpha}{2}=1.96$

$\because$
$\varDelta=Z_\frac{\alpha}{2}\dfrac{S}{\sqrt{n}}=1.96\times \dfrac{6}{\sqrt{20}}\approx2.63$

$\therefore$全校学生身高均值的95\%的置信区间是$171\pm2.63$,即$(168.37,173.63)$
%\end{spacing}


\paragraph{3 从某正态总体中随机抽取样本容量为25的样本,该样本的分散程度$S=10$,问该总体的分散程度如何?}
~\\

%\begin{spacing}{2.0}
给定显著性水平为$\alpha=0.05$,则$\chi^2_{0.025}(24)=39.364,\chi^2_{0.975}(24)=12.401$
~\\

总体方差的95\%的置信区间的下界为$\dfrac{(n-1)S^2}{\chi^2_{0.975}(24)}=\dfrac{24\times 100}{12.401}\approx193.53$
~\\

总体方差的95\%的置信区间的上界为$\dfrac{(n-1)S^2}{\chi^2_{0.025}(24)}=\dfrac{24\times 100}{39.364}\approx60.97$
~\\

$\sigma^2 \in (193.53,60.97) \quad \Rightarrow \quad \sigma \in(7.81,13.91)$

$\therefore$总体方差的95\%的置信区间是$(193.53,60.97)$,标准差95\%的置信区间是$(7.81,13.91)$
%\end{spacing}


\paragraph{4 从两个正态总体中各随机抽取一个样本,$n_1=10, S_1=3; n_2=11, S_2=4$,求二总体方差比的95\%的置信区间。}
~\\

%\begin{spacing}{2.0}
给定显著性水平为$\alpha=0.1$,则$F_{0.05}(9,10)=3.02,F_{0.05}(10,9)=3.14$
~\\

总体方差比$\dfrac{\sigma_1^2}{\sigma_2^2}$的95\%的置信区间的下界是$\dfrac{S_1^2}{S_2^2}\times \dfrac{1}{F_{0.05}(9,10)}=\dfrac{9}{16}\times \dfrac{1}{3.02}\approx0.19$
~\\

总体方差比$\dfrac{\sigma_1^2}{\sigma_2^2}$的95\%的置信区间的上界是$\dfrac{S_1^2}{S_2^2}\times F_{0.05}(10,9)=\dfrac{9}{16}\times 3.14\approx1.77$
~\\

$\therefore$总体方差比的95\%的置信区间是$(0.19,1.77)$
%\end{spacing}


\paragraph{5 已知样本相关系数$r=0.60, n=37$,问总体相关系数是多少?}
~\\

%\begin{spacing}{2.0}
给定显著性水平为$\alpha=0.05$,则$Z_\frac{\alpha}{2}=1.96$

$Z_r=\dfrac{1}{2}\cdot\ln\dfrac{1+r}{1-r}=\dfrac{1}{2}\cdot\ln\dfrac{1.6}{0.4}=\ln2\approx0.6931$

$SE_r=\dfrac{1}{\sqrt{n-3}}=\dfrac{1}{\sqrt{34}}$

$\varDelta=Z_\frac{\alpha}{2}SE_r=1.96\times \dfrac{1}{\sqrt{34}}\approx0.3361$

$\therefore$
$Z_r\in(0.3570,1.0292)$

$\because$
$r=\dfrac{e^{2Z_r}-1}{e^{2Z_r}+1} \quad \Rightarrow \quad r\in(0.34,0.77)$

$\therefore$总体相关系数的95\%的置信区间是$(0.34,0.77)$
%\end{spacing}

\paragraph{6 某县教育局随机抽查了360名初中学生的实力情况,发现有125名学生患有不同程度的近视,问该县初中学生患近视的真实比例是多少?}
~\\

%\begin{spacing}{2.0}
给定显著性水平为$\alpha=0.05$,则$Z_\frac{\alpha}{2}=1.96$

$\because$
样本的比例$\hat{p}=\dfrac{125}{360}\approx0.3472$

$\varDelta=Z_\frac{\alpha}{2}\dfrac{\sqrt{\hat{p}(1-\hat{p})}}{\sqrt{n}}=1.96\times \dfrac{\sqrt{125\times 235}}{360\times \sqrt{360}}\approx0.0156$

$\therefore$患近视的真实比例的95\%的置信区间是$0.3472\pm0.0156$,即$(0.3361,0.3628)$
%\end{spacing}


\paragraph{7 从$\sigma^2=25$的正态总体中抽取$n=10$的样本为10、20、17、19、25、24、22、31、26、26,求$\chi^2$及大于该值的概率。但如果$\mu=23$,其$\chi^2$值以及大于该值的概率又是多少?}
~\\

%\begin{spacing}{2.0}
$\Bar{X}=\dfrac{10+20+17+19+25+24+22+31+26+26}{10}=\dfrac{220}{10}=22$

$S^2=\dfrac{\displaystyle \sum_{i=1}^n (X_i-\Bar{X})^2}{n-1}=\dfrac{308}{9}$
~\\

因为$\dfrac{(n-1)S^2}{\sigma^2} \sim \chi^2(n-1)$,记自由度为9时卡方分布的概率密度函数为$f(x_9)$
~\\

$\therefore \chi^2=\dfrac{(n-1)S^2}{\sigma^2}=\dfrac{9\times 308}{25\times 9}=12.32; 
\quad P(\chi^2>12.32)=\int_{12.32}^{+\infty}f(x_9)\mathrm{d}x_9=0.1959$
~\\

如果$\mu=23$,因为$\dfrac{\displaystyle \sum_{i=1}^n (X_i-\mu)^2}{\sigma^2} \sim \chi^2(n)$,记自由度为10时卡方分布的概率密度函数为$f(x_{10})$
~\\

%=\dfrac{\displaystyle \sum_{i=1}^n (X_i-23)^2}{25}
$\therefore \chi^2=\dfrac{\displaystyle \sum_{i=1}^n (X_i-\mu)^2}{\sigma^2}=\dfrac{318}{25}=12.72;
\quad P(\chi^2>12.72)=\int_{12.72}^{+\infty}f(x_{10})\mathrm{d}x_{10}=0.2398$

%\end{spacing}


\end{document}
