\documentclass[UTF8]{ctexart}
\usepackage{ctex,geometry}
\usepackage{mathrsfs}
\usepackage{amsmath}
\usepackage{amssymb}
\usepackage{booktabs}
\pagestyle{plain}
\geometry{left=2.5cm,right=2.5cm,top=2.0cm,bottom=2cm}
\begin{document}

\begin{center}
	\LARGE
	2019级心理多元统计课程作业\_第五次

	\normalsize
	谷菲 201928012503005
\end{center}

\paragraph{1地鼠体重}
~\\

记地鼠七月份体重的总体均值为$\mu_7$

$H_0:\mu=\mu_7 \qquad H_1:\mu\ne\mu_7$

$Z=\dfrac{\bar{X}-\mu}{\dfrac{\sigma}{\sqrt{n}}}=\dfrac{350-400}{\dfrac{100}{\sqrt{25}}}=-2.5$

当双侧检验$\alpha=0.05$时,临界值$Z_{\frac{\alpha}{2}}=1.96$

$\because |Z| > 1.96$

$\therefore$拒绝$H_0$,七月份和八月份的体重有显著差异

\paragraph{2麻醉药生效时间}
~\\

记新麻醉药的生效时间的均值为$\mu_1$

$H_0:\mu_1\ge\mu \qquad H_1:\mu_1<\mu$
~\\

$\bar{X}=\dfrac{9.3+9.5+9.2+9.0+9.3+9.5+9.4+9.3+9.2+9.1}{10}=9.27$

$S^2=\dfrac{\displaystyle \sum_{i=1}^n (X_i-\Bar{X})^2}{n-1}=\dfrac{0.281}{9}\approx0.03$
~\\

$S=\sqrt{S^2}\approx0.18$
~\\

$Z=\dfrac{\bar{X}-\mu}{\dfrac{S}{\sqrt{n}}}=\dfrac{9.27-10.5}{\dfrac{0.18}{\sqrt{10}}}=-21.61$
~\\
	
当单尾检验$\alpha=0.05$时,临界值$Z_{\alpha}=1.645$

$\because |Z| > 1.645$
	
$\therefore$拒绝$H_0$,新麻醉药的生效时间显著快于原来的

\paragraph{3就餐平均次数}
~\\

$H_0:\mu_1=\mu_2 \qquad H_1:\mu_1\ne\mu_2$
~\\

虽然总体不服从正态分布,但样本量较大
~\\

$Z=\dfrac{\bar{X_1}-\bar{X_2}}{\sqrt{\dfrac{S_1^2}{n_1}+\dfrac{S_2^2}{n_2}}}=\dfrac{6.5-5.8}{\sqrt{\dfrac{2.1^2}{120}+\dfrac{1.8^2}{120}}}=2.77$
~\\

当双侧检验$\alpha=0.05$时,临界值$Z_{\frac{\alpha}{2}}=1.96$

$\because Z > 1.96$

$\therefore$拒绝$H_0$,两个社区的人到餐馆就餐的平均次数有显著差异

\paragraph{4深度知觉实验}
~\\

记实验组的误差均值为$\mu_1$,控制组的均值为$\mu_2$

$H_0:\mu_1\ge\mu_2 \qquad H_1:\mu_1<\mu_2$
~\\

$t=\dfrac{\bar{X_1}-\bar{X_2}}{\sqrt{\dfrac{S_1^2}{n_1}+\dfrac{S_2^2}{n_2}}}=\dfrac{4-6.5}{\sqrt{\dfrac{2^2}{12}+\dfrac{2.5^2}{12}}}=-2.71$
~\\

$df=n_1+n_2-2=24-2=22$

当单侧检验$\alpha=0.05$时,临界值$t_{\alpha}(22)=1.717$

$\because |t| > 1.717$

$\therefore$拒绝$H_0$,训练显著减少了深度知觉的误差

\paragraph{5双生子相关系数}
~\\

记同卵双生子相关为$\rho_1$,异卵双生子相关为$\rho_2$

$H_0:\rho_1=\rho_2 \qquad H_1:\rho_1\ne\rho_2$
~\\

$Z_{r1}=\dfrac{1}{2}\cdot\ln\dfrac{1+r_1}{1-r_1}=\dfrac{1}{2}\cdot\ln\dfrac{1.85}{0.15}\approx1.256 \qquad Z_{r2}=\dfrac{1}{2}\cdot\ln\dfrac{1+r_2}{1-r_2}=\dfrac{1}{2}\cdot\ln\dfrac{1.76}{0.24}\approx0.996$
~\\

%$SE_{r1}=\dfrac{1}{\sqrt{n_1-3}}=\dfrac{1}{\sqrt{14}}\approx0.267 \qquad SE_{r2}=\dfrac{1}{\sqrt{n_2-3}}=\dfrac{1}{\sqrt{21}}\approx0.218$ 

$Z=\dfrac{Z_{r1}-Z_{r2}}{\sqrt{\dfrac{1}{n_{1}-3}+\dfrac{1}{n_{2}-3}}}=\dfrac{1.256-0.996}{\sqrt{\dfrac{1}{14}+\dfrac{1}{21}}}\approx0.66$
~\\

当双侧检验$\alpha=0.05$时,临界值$Z_{\frac{\alpha}{2}}=1.96$

$\because Z < 1.96$

$\therefore$接受$H_0$,同卵双生子和异卵双生子的相关系数没有显著差异

%$r=\dfrac{e^{2Z_r}-1}{e^{2Z_r}+1}$

\paragraph{6健美俱乐部}
~\\

\begin{tabular}{cccc}
\toprule
训练前 & 训练后 & 减少值$X_i$ & $(X_i-\bar{X})^2$ \\
\midrule
94.5 & 85 & 9.5 & 0.0225 \\
101 & 89.5 & 11.5 & 3.4225 \\
110 & 101.5 & 8.5 & 1.3225 \\
103.5 & 96 & 7.5 & 4.6225 \\
97 & 86 & 11 & 1.8225 \\
88.5 & 80.5 & 8 & 2.7225 \\
96.5 & 87 & 9.5 & 0.0225 \\
101 & 95.5 & 5.5 & 17.2225 \\
104 & 93 & 11 & 1.8225 \\
116.5 & 102 & 14.5 & 23.5225 \\

\bottomrule
\end{tabular}

记减少体重的均值为$\mu$

$H_0:\mu<8.5 \qquad H_1:\mu\ge8.5$


$\bar{X}=\dfrac{\displaystyle \sum_{i=1}^{n}X_i}{n}=\dfrac{96.5}{10}=9.65$
~\\

$S^2=\dfrac{\displaystyle \sum_{i=1}^n (X_i-\Bar{X})^2}{n-1}=\dfrac{56.525}{9}\approx6.28$
~\\

$S=\sqrt{6.28}=2.51$
~\\

$t=\dfrac{\bar{X}-\mu}{\dfrac{\sigma}{\sqrt{n}}}=\dfrac{9.65-8.5}{\dfrac{2.51}{\sqrt{10}}}\approx1.45$
~\\

$df=n-1=10-1=9$

当单侧检验$\alpha=0.05$时,临界值$t_{\alpha}(9)=1.833$

$\because t < 1.833$

$\therefore$接受$H_0$,减少的重量没有显著多于8.5kg

\paragraph{7方差的检验}
~\\

$H_0:\sigma^2=2 \qquad H_1:\sigma^2\ne2$


$\bar{X}=\dfrac{\displaystyle \sum_{i=1}^{n}X_i}{n}=\dfrac{11+7+2+9+3}{5}=6.4$

$S^2=\dfrac{\displaystyle \sum_{i=1}^n (X_i-\Bar{X})^2}{n-1}=\dfrac{4.6^2+0.6^2+4.4^2+2.6^2+3.4^2}{4}=14.8$
~\\

%$S=\sqrt{S^2}=\sqrt{14.8}\approx3.85$
%~\\

$\chi^2=\dfrac{(n-1)S^2}{\sigma^2}=\dfrac{4\times 14.8}{2}=29.6$

$df=n-1=4$

当$\alpha=0.05$时,临界值$\chi^2_{\frac{\alpha}{2}}(4)=0.48 \qquad \chi^2_{1-\frac{\alpha}{2}}(4)=11.14$

$\because \chi^2 > 11.14$

$\therefore$拒绝$H_0$,$\sigma^2$和2有显著差异

\paragraph{8可代谢能量}
~\\

记喂罐头猫食的方差为$\sigma_1$,喂干猫食的方差为$\sigma_2$

$H_0:\sigma_1^2=\sigma_2^2 \qquad H_1:\sigma_1^2\ne\sigma_2^2$

$S_1^2=0.26^2=0.0676 \qquad S_2^2=0.48^2=0.2304$

$F=\dfrac{S_2^2}{S_1^2}=\dfrac{0.2304}{0.0676}\approx3.41$

当$\alpha=0.05$时,临界值$F_{\frac{\alpha}{2}}(28,27)=2.15$

$\because F > 2.15$

$\therefore$拒绝$H_0$,两种喂养方式的猫的可代谢能量成分的方差有显著差异

\paragraph{9杀虫剂}
~\\

记杀虫剂可以杀死的蟑螂的比例为$p$,广告宣传的比例为$p_0=0.95$

$H_0:p\le p_0 \qquad H_1:p > p_0$
~\\

$\hat{p}=\dfrac{384}{400}=0.96$
~\\

$Z=\dfrac{\hat{p}-p_0}{\sqrt{\dfrac{p_0(1-p_0)}{n}}}=\dfrac{0.96-0.95}{\sqrt{\dfrac{0.95\times 0.05}{400}}}\approx0.92$

当单侧检验$\alpha=0.05$时,临界值$Z_\alpha=1.645$

$\because Z < 1.645$

$\therefore$接受$H_0$,没有充分的证据支持广告中的说法

\paragraph{10左右利手}
~\\

记左利手中因伤住院的比例为$p_1$,右利手中因伤住院的比例为$p_2$

$H_0:p_1 \le p_2 \qquad H_1:p_1 > p_2$
~\\

$\hat p_1=\dfrac{90}{231}\approx0.39 \qquad \hat p_2=\dfrac{623}{2148}\approx0.29$
~\\

$p_e=\dfrac{n_1\hat p_1+n_2\hat p_2}{n_1+n_2}=\dfrac{90+623}{231+2146}\approx0.30$
~\\

$Z=\dfrac{\hat p_1 - \hat p_2}{\sqrt{p_eq_e\left(\dfrac{1}{n_1}+\dfrac{1}{n_2}\right)}}=\dfrac{0.39-0.29}{\sqrt{0.3\times0.7\times\left(\dfrac{1}{231}+\dfrac{1}{2146}\right)}}\approx3.15$
~\\

当单侧检验$\alpha=0.01$时,临界值$Z_\alpha=2.33$

$\because Z > 2.33$

$\therefore$拒绝$H_0$,左撇子出事故的比例显著高于右撇子

\end{document}