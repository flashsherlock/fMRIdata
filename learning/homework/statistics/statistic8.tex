\documentclass[UTF8]{ctexart}
\usepackage{ctex,geometry}
\usepackage{mathrsfs}
\usepackage{amsmath}
\usepackage{amssymb}
\usepackage{booktabs,multirow}
\pagestyle{plain}
\geometry{left=2.5cm,right=2.5cm,top=2.0cm,bottom=2cm}
\begin{document}

\begin{center}
	\LARGE
	2019级心理多元统计课程作业\_第八次

	\normalsize
	谷菲 201928012503005
\end{center}

\paragraph{1求相关矩阵}
~\\

%\[ \mathbf{X} = \left(
%\begin{array}{cccc}
%x_{11} & x_{12} & \ldots & x_{1n}\\
%x_{21} & x_{22} & \ldots & x_{2n}\\
%\vdots & \vdots & \ddots & \vdots\\
%x_{n1} & x_{n2} & \ldots & x_{nn}\\
%\end{array} \right) \]
由于协方差矩阵
$
\Sigma=
\begin{pmatrix}
	25 & $-$2 & 4\\
	$-$2 & 4 & 1\\
	4 & 1 & 9\\
\end{pmatrix}
$
,所以标准差矩阵为
$
\mathbf{D}=
\begin{pmatrix}
5 & 0 & 0\\
0 & 2 & 0\\
0 & 0 & 3\\
\end{pmatrix}
$
\\

相关矩阵
$
\rho=\mathbf{D^{-1}} \Sigma \mathbf{D^{-1}}=
\begin{pmatrix}
\frac{1}{5} & 0 & 0\\
0 & \frac{1}{2} & 0\\
0 & 0 & \frac{1}{3} \\
\end{pmatrix}
\begin{pmatrix}
25 & $-$2 & 4\\
$-$2 & 4 & 1\\
4 & 1 & 9\\
\end{pmatrix}
\begin{pmatrix}
\frac{1}{5} & 0 & 0\\
0 & \frac{1}{2} & 0\\
0 & 0 & \frac{1}{3} \\
\end{pmatrix}
=
\begin{pmatrix}
1 & $-$\frac{1}{5} & \frac{4}{15}\\
$-$\frac{1}{5} & 1 & \frac{1}{6}\\
\frac{4}{15} & \frac{1}{6} & 1\\
\end{pmatrix}
$

\paragraph{2求协方差矩阵}
~\\

由各个分量的方差可得标准差矩阵
$
\mathbf{D}=
\begin{pmatrix}
4 & 0 & 0& 0\\
0 & 1 & 0& 0\\
0 & 0 & 3& 0\\
0 & 0 & 0& 2\\
\end{pmatrix}
$
\\


$
\begin{aligned}
\Sigma=\mathbf{D} \rho \mathbf{D}
&=
\begin{pmatrix}
4 & 0 & 0& 0\\
0 & 1 & 0& 0\\
0 & 0 & 3& 0\\
0 & 0 & 0& 2\\
\end{pmatrix}
\begin{pmatrix}
1& $-$0.1& 0.25& $-$0.4\\
$-$0.1& 1& 0.6& 0.8\\
0.25& $-$0.6& 1& 0.2\\
$-$0.4& 0.8& 0.2& 1\\
\end{pmatrix}
\begin{pmatrix}
4 & 0 & 0& 0\\
0 & 1 & 0& 0\\
0 & 0 & 3& 0\\
0 & 0 & 0& 2\\
\end{pmatrix}\\
&=
\begin{pmatrix}
4& $-$0.4& 1& $-$1.6\\
$-$0.1& 1& 0.6& 0.8\\
0.75& $-$1.8& 3&0.6 \\
$-$0.8& 1.6& 0.4& 2\\
\end{pmatrix}
\begin{pmatrix}
4 & 0 & 0& 0\\
0 & 1 & 0& 0\\
0 & 0 & 3& 0\\
0 & 0 & 0& 2\\
\end{pmatrix}\\
&=
\begin{pmatrix}
16& $-$0.4& 3& $-$3.2\\
$-$0.4& 1& 1.8& 1.6\\
3& $-$1.8& 9& 1.2 \\
$-$3.2& 1.6& 1.2& 4\\
\end{pmatrix}
\end{aligned}
$


\end{document}